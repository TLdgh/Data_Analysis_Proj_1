% Options for packages loaded elsewhere
\PassOptionsToPackage{unicode}{hyperref}
\PassOptionsToPackage{hyphens}{url}
%
\documentclass[
]{article}
\usepackage{amsmath,amssymb}
\usepackage{iftex}
\ifPDFTeX
  \usepackage[T1]{fontenc}
  \usepackage[utf8]{inputenc}
  \usepackage{textcomp} % provide euro and other symbols
\else % if luatex or xetex
  \usepackage{unicode-math} % this also loads fontspec
  \defaultfontfeatures{Scale=MatchLowercase}
  \defaultfontfeatures[\rmfamily]{Ligatures=TeX,Scale=1}
\fi
\usepackage{lmodern}
\ifPDFTeX\else
  % xetex/luatex font selection
\fi
% Use upquote if available, for straight quotes in verbatim environments
\IfFileExists{upquote.sty}{\usepackage{upquote}}{}
\IfFileExists{microtype.sty}{% use microtype if available
  \usepackage[]{microtype}
  \UseMicrotypeSet[protrusion]{basicmath} % disable protrusion for tt fonts
}{}
\makeatletter
\@ifundefined{KOMAClassName}{% if non-KOMA class
  \IfFileExists{parskip.sty}{%
    \usepackage{parskip}
  }{% else
    \setlength{\parindent}{0pt}
    \setlength{\parskip}{6pt plus 2pt minus 1pt}}
}{% if KOMA class
  \KOMAoptions{parskip=half}}
\makeatother
\usepackage{xcolor}
\usepackage[margin=1in]{geometry}
\usepackage{longtable,booktabs,array}
\usepackage{calc} % for calculating minipage widths
% Correct order of tables after \paragraph or \subparagraph
\usepackage{etoolbox}
\makeatletter
\patchcmd\longtable{\par}{\if@noskipsec\mbox{}\fi\par}{}{}
\makeatother
% Allow footnotes in longtable head/foot
\IfFileExists{footnotehyper.sty}{\usepackage{footnotehyper}}{\usepackage{footnote}}
\makesavenoteenv{longtable}
\usepackage{graphicx}
\makeatletter
\def\maxwidth{\ifdim\Gin@nat@width>\linewidth\linewidth\else\Gin@nat@width\fi}
\def\maxheight{\ifdim\Gin@nat@height>\textheight\textheight\else\Gin@nat@height\fi}
\makeatother
% Scale images if necessary, so that they will not overflow the page
% margins by default, and it is still possible to overwrite the defaults
% using explicit options in \includegraphics[width, height, ...]{}
\setkeys{Gin}{width=\maxwidth,height=\maxheight,keepaspectratio}
% Set default figure placement to htbp
\makeatletter
\def\fps@figure{htbp}
\makeatother
\setlength{\emergencystretch}{3em} % prevent overfull lines
\providecommand{\tightlist}{%
  \setlength{\itemsep}{0pt}\setlength{\parskip}{0pt}}
\setcounter{secnumdepth}{-\maxdimen} % remove section numbering
\newenvironment{cols}[1][]{}{}

\newenvironment{col}[1]{\begin{minipage}{#1}\ignorespaces}{%
\end{minipage}
\ifhmode\unskip\fi
\aftergroup\useignorespacesandallpars}

\def\useignorespacesandallpars#1\ignorespaces\fi{%
#1\fi\ignorespacesandallpars}

\makeatletter
\def\ignorespacesandallpars{%
  \@ifnextchar\par
    {\expandafter\ignorespacesandallpars\@gobble}%
    {}%
}
\makeatother
\renewcommand{\and}{\\}
\usepackage{booktabs}
\usepackage{longtable}
\usepackage{array}
\usepackage{multirow}
\usepackage{wrapfig}
\usepackage{float}
\usepackage{colortbl}
\usepackage{pdflscape}
\usepackage{tabu}
\usepackage{threeparttable}
\usepackage{threeparttablex}
\usepackage[normalem]{ulem}
\usepackage{makecell}
\usepackage{xcolor}
\ifLuaTeX
  \usepackage{selnolig}  % disable illegal ligatures
\fi
\IfFileExists{bookmark.sty}{\usepackage{bookmark}}{\usepackage{hyperref}}
\IfFileExists{xurl.sty}{\usepackage{xurl}}{} % add URL line breaks if available
\urlstyle{same}
\hypersetup{
  pdftitle={MAT5314 Project 1: Data Visualization},
  pdfauthor={Teng Li(7373086); Shiya Gao(300381032); Chuhan Yue(300376046); Yang Lyu(8701121)},
  hidelinks,
  pdfcreator={LaTeX via pandoc}}

\title{MAT5314 Project 1: Data Visualization}
\author{Teng Li(7373086) \and Shiya Gao(300381032) \and Chuhan
Yue(300376046) \and Yang Lyu(8701121)}
\date{}

\begin{document}
\maketitle

\hypertarget{introduction}{%
\section{Introduction}\label{introduction}}

A data set of the 2016 US election polls was given. In this project we
aim to understand the data structure by creating various visualizations.

First, we would like to give a brief introduction to the U.S. election
system, because it is crucial to understand the background of the data.
Voters in each state vote to choose the President of the United States.
The candidate who wins the majority of the votes will receive all the
electoral votes in that state. Then the sum of the electoral votes in
each state is calculated. The total number of electoral votes is 538.
The candidate who wins half of the votes plus 1 will win and become the
new President of the United States.

Secondly, we want to analyze the key factors for the candidate's
victory. Since each state has a different number of electoral votes, it
is crucial for electors to win in several key states. The reason is that
if a candidate wins a certain state, he will win all the electoral votes
in that state. So there will be tight competition in states with more
votes.

The data set was published by FiveThirtyEight to illustrate the
reliability and quality of each pollster to which a letter grade ranging
from A+ to D- was given.

\hypertarget{method}{%
\section{Method}\label{method}}

We use various R packages to present the data set and to plot the
graphs.

\hypertarget{result}{%
\section{Result}\label{result}}

We first created a data variable definition table to give an initial
understanding of the data. As one can see, there were a few variables
with missing values:

\begin{table}[!h]

\caption{\label{tab:unnamed-chunk-3}Data Variable Definition}
\centering
\resizebox{\linewidth}{!}{
\fontsize{10}{12}\selectfont
\begin{tabular}[t]{lrl>{\raggedright\arraybackslash}p{8em}rrl}
\toprule
\textbf{Variables} & \textbf{Size} & \textbf{Type} & \textbf{Example} & \textbf{Number.Unique} & \textbf{Number.Missing} & \textbf{Comment}\\
\midrule
\cellcolor{gray!6}{state} & \cellcolor{gray!6}{4208} & \cellcolor{gray!6}{character} & \cellcolor{gray!6}{U.S., New Mexico, Virginia} & \cellcolor{gray!6}{57} & \cellcolor{gray!6}{0} & \cellcolor{gray!6}{The name of the state (or national) where the election is held}\\
startdate & 4208 & character & 2016/11/3, 2016/11/1, 2016/11/2 & 352 & 0 & Start date of poll\\
\cellcolor{gray!6}{enddate} & \cellcolor{gray!6}{4208} & \cellcolor{gray!6}{character} & \cellcolor{gray!6}{2016/11/6, 2016/11/7, 2016/11/5} & \cellcolor{gray!6}{345} & \cellcolor{gray!6}{0} & \cellcolor{gray!6}{End date of poll}\\
pollster & 4208 & character & ABC News/Washington Post, Google Consumer Surveys, Ipsos & 196 & 0 & Organization name that conducts or analyzes opinion polls\\
\cellcolor{gray!6}{grade} & \cellcolor{gray!6}{4208} & \cellcolor{gray!6}{character} & \cellcolor{gray!6}{A+, B, A-} & \cellcolor{gray!6}{11} & \cellcolor{gray!6}{429} & \cellcolor{gray!6}{Grade assigned by Fivethirtyeight to pollster}\\
\addlinespace
samplesize & 4208 & integer & 2220, 26574, 2195 & 1767 & 1 & Sample size of polls for each pollster\\
\cellcolor{gray!6}{population} & \cellcolor{gray!6}{4208} & \cellcolor{gray!6}{character} & \cellcolor{gray!6}{lv, rv, a} & \cellcolor{gray!6}{4} & \cellcolor{gray!6}{0} & \cellcolor{gray!6}{Type of population being polled}\\
rawpoll\_clinton & 4208 & numeric & 47, 38.03, 42 & 1312 & 0 & Poll Percentage for Hillary Clinton\\
\cellcolor{gray!6}{rawpoll\_trump} & \cellcolor{gray!6}{4208} & \cellcolor{gray!6}{numeric} & \cellcolor{gray!6}{43, 35.69, 39} & \cellcolor{gray!6}{1385} & \cellcolor{gray!6}{0} & \cellcolor{gray!6}{Poll Percentage for Donald Trump}\\
rawpoll\_johnson & 4208 & numeric & 4, 5.46, 6 & 585 & 1409 & Poll Percentage for Gary Johnson\\
\addlinespace
\cellcolor{gray!6}{rawpoll\_mcmullin} & \cellcolor{gray!6}{4208} & \cellcolor{gray!6}{numeric} & \cellcolor{gray!6}{NA, 24, 27.6} & \cellcolor{gray!6}{17} & \cellcolor{gray!6}{4178} & \cellcolor{gray!6}{Poll Percentage for Evan Mcmullin}\\
adjpoll\_clinton & 4208 & numeric & 45.20163, 43.34557, 42.02638 & 4200 & 0 & Adjusted percentage for Hillary Clinton\\
\cellcolor{gray!6}{adjpoll\_trump} & \cellcolor{gray!6}{4208} & \cellcolor{gray!6}{numeric} & \cellcolor{gray!6}{41.7243, 41.21439, 38.8162} & \cellcolor{gray!6}{4204} & \cellcolor{gray!6}{0} & \cellcolor{gray!6}{Adjusted percentage for Donald Trump}\\
adjpoll\_johnson & 4208 & numeric & 4.626221, 5.175792, 6.844734 & 2211 & 1409 & Adjusted percentage for Gary Johnson\\
\cellcolor{gray!6}{adjpoll\_mcmullin} & \cellcolor{gray!6}{4208} & \cellcolor{gray!6}{numeric} & \cellcolor{gray!6}{NA, 24, 27.70142} & \cellcolor{gray!6}{31} & \cellcolor{gray!6}{4178} & \cellcolor{gray!6}{Adjusted percentage for Evan Mcmullin}\\
\bottomrule
\end{tabular}}
\end{table}

Note that the poll results for Johnson and McMullin had lots of missing
values. In particular, Johnson had 33.48\% raw poll result and 33.48\%
adjusted poll result missing, and McMullin had 99.29\% and 99.29\%
missing. Due to the fact that these two candidate didn't make to the
final election, we chose to ignore their data in some of the analysis.

Since the final two candidates in Election 2016 are Clinton and Trump,
we plotted box plots of the difference of their poll results, one for
the raw poll and one for the adjusted poll. We saw that there's little
difference between the distribution of the raw and the adjusted data.
However, the mean of each grade of the adjusted poll result was a little
closer to zero than that of the raw poll result. This indicated that the
adjustment that FiveThirtyEight made was an improvement because the raw
poll difference of each grade was mostly above zero, which clearly
showed that the poll result was more in favour of Clinton yet Trump was
the final winner of the election.\\
\includegraphics{MAT5314-Project-1_files/figure-latex/unnamed-chunk-4-1.png}

\begin{cols}

\begin{col}{0.55\textwidth}
We notice that there are 57 pollsters (almost 30\% of the total number
of pollsters) whose grades are missing in this data set. And we cannot
just delete them, because it will cause a lot of missing data in other
columns. We suppose that there are two possible reasons for these
missing data: one is that there are some errors of data in the original
file; the other is that fivethirtyeight has not rated these pollsters
yet. So we searched online and found a more detailed and authoritative
file about the pollsters' grade from the fivethirtyeight website. Here
is the link:
\url{https://projects.fivethirtyeight.com/pollster-ratings/}.

\vspace{18pt}

According to the fivethirtyeight website, we found that 26 pollsters
with no grade in the origin data set actually have the grades like
``A/B'', ``B/C'', ``C/D'', ``B'', ``B-''; otherwise, the rest 31
pollsters without grades haven't been scored yet. Based on these
information, we updated the ``grade'' column of the origin data set. We
replace ``NA'' with the actual grades and none.

\end{col}

\begin{col}{0.05\textwidth}
~

\end{col}

\begin{col}{0.4\textwidth}

\begin{longtable}[]{@{}llc@{}}
\caption{Updating the missing data}\tabularnewline
\toprule\noalign{}
& pollster & grade \\
\midrule\noalign{}
\endfirsthead
\toprule\noalign{}
& pollster & grade \\
\midrule\noalign{}
\endhead
\bottomrule\noalign{}
\endlastfoot
23 & Remington & B \\
27 & Morning Consult & B- \\
35 & Saguaro Strategies & B/C \\
37 & Insights West & B/C \\
44 & BK Strategies & B/C \\
59 & Data Orbital & A/B \\
65 & Starboard Communications & B/C \\
69 & Strategic National & B/C \\
84 & Bendixen \& Amandi International & B/C \\
86 & Associated Industries of Florida & B/C \\
97 & Centre College & B/C \\
98 & Public Religion Research Institute & A/B \\
101 & Praecones Analytica & B/C \\
106 & Craciun Research & B/C \\
108 & University of Colorado & B/C \\
112 & Baldwin Wallace University & B/C \\
122 & University of Wyoming & C/D \\
131 & HighGround & B/C \\
133 & Michigan State University & A/B \\
140 & Echelon Insights & A/B \\
152 & Meredith College & B/C \\
155 & Mercyhurst University & B/C \\
167 & Strategy Research & B/C \\
176 & Hickman Analytics & B/C \\
184 & Data Targeting & B/C \\
196 & Ogden \& Fry & B/C \\
\end{longtable}

\end{col}

\end{cols}

\vspace{18pt}

Because some pollsters are repeated in different rows of the data set,
so we want to verify that each pollster corresponds to only one kind of
grade. The result is as follow:

\begin{verbatim}
## the column of number_of_grades only contains one type of value: 1
\end{verbatim}

From the ``Pie Chart of Pollsters' Grades'' below, we could see that the
unrated pollsters make up the largest percentage, about 15.8\% of the
total; pollsters with grades ``B'' and ``C+'' account for the second and
third most, 12.8\% and 12.2\% respectively; ``D'' grade has the smallest
percentage of pollsters, only about 0.51\%. Besides, B-level grades,
including ``B+'', ``B'' and ``B-'', are around 34.7\%, almost one-third
of the total; C-level and A-level grades are about 21.89\% and 14.79\%
respectively. For those without explicit grades, whose grades are
``A/B'', ``B/C'' and ``C/D'', they account only for 12.24\%, ``B'' grade
pollsters make up the majority of this part especially, almost 9.69\%.

\includegraphics{MAT5314-Project-1_files/figure-latex/unnamed-chunk-7-1.png}

\hypertarget{polls-trend-by-time}{%
\subsection{Polls trend by time}\label{polls-trend-by-time}}

Using the end date as the standard, we drew a scatter plot of the
changes in the support rates of the four candidates over time.

\begin{figure}
\centering
\includegraphics{MAT5314-Project-1_files/figure-latex/unnamed-chunk-9-1.pdf}
\caption{Raw Poll Proportion by Time.}
\end{figure}

Figure 1 indicates that Clinton and Trump's approval ratings are
significantly higher than those of Johnson and Mcmullin at the overall
level. Another interesting demonstration is the divergence of support as
the vote draws to a close. This shows that as the voting deadline
approaches, people's intentions are more inclined to one of Clinton or
Trump. Voting is also more polarized. A low poll rate for one candidate
also implies that other candidates may have received high poll rates.
This creates differences in pollsters or states.

\hypertarget{comparison-of-candidates-polls-in-each-state}{%
\subsection{Comparison of candidates' polls in each
state}\label{comparison-of-candidates-polls-in-each-state}}

We want to process the metadata by counting the polls received by each
of the four candidates in each state. This result is easier to obtain by
multiplying the given size and the proportion. Regardless of the various
pollsters, we combine the number of polls for each candidate received in
each state although the polls may come from different pollsters. In this
case, we treat NA as 0.

The visualization of the poll proportion of the four candidates in each
state as follows.

\begin{figure}
\centering
\includegraphics{MAT5314-Project-1_files/figure-latex/unnamed-chunk-14-1.pdf}
\caption{Poll Percentage by States.}
\end{figure}

Figure 2 shows the poll proportions of the four candidates in each state
distinguished by colors. We can clearly observe which candidate is
likely to win all the electoral votes in each state, which is helpful in
estimating the outcome of the presidential election. From the chart
above, we conclude that Clinton and Trump are clearly ahead of Johnson
and Mcmullin. Furthermore, Clinton is clearly ahead of Trump in
California, DC, Hawaii, Illinois, Maryland, Massachusetts, New Jersey,
New York, Oregon, Rhode Island, Vermont, and Washington. Another side,
in Alabama, Alaska, Idaho, Indiana, Iowa, Kansas, Kentucky, Louisiana,
Mississippi, Missouri, Montana, Nebraska, North Dakota, Ohio, Oklahoma,
South Carolina, South Dakota, Tennessee, Texas, Utah, West Virginia,
Wyoming, Trump clearly leads Clinton. This helps us tally who would win
all of the state's electoral votes.

To make state-by-state polling results more visible, we've reflected the
candidates' polling shares on a map of the United States. Since the
other candidates are polling far behind Clinton and Trump, we select the
Clinton and Trump polls and ignore the other candidates' polls. In the
contour plot below, blue indicates that Clinton is polling
proportionally greater than Trump, and red indicates that Trump is
polling proportionally greater than Clinton. The darker the color, the
more overwhelming the proportions. For states that are nearly white, the
margin of victory is nearly half between Trump and Clinton.

\includegraphics{MAT5314-Project-1_files/figure-latex/unnamed-chunk-19-1.png}

\hypertarget{discussion}{%
\section{Discussion}\label{discussion}}

\hypertarget{conclusion}{%
\section{Conclusion}\label{conclusion}}

\end{document}
